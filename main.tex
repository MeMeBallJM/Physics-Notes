\documentclass{article}
\usepackage[a4paper, total={6in, 8in}]{geometry}


% ams packages (usefull for a bunch of stuff)
\usepackage{amsmath}
\usepackage{amsthm}
\usepackage{amssymb}
\usepackage{mathrsfs}

% tikz (creates drawings)
\usepackage{tikz}
\usetikzlibrary{arrows.meta}

% tcolorbox (draws colored boxes)
\usepackage[most,many,breakable]{tcolorbox}

% Add colors
\usepackage{xcolor}

% Links
\usepackage{hyperref}

% Variable with objects (tbh I don't really know)
\usepackage{varwidth}


\usepackage[framemethod=TikZ]{mdframed}



% Bold vectors instead of arrow
\let\vec\mathbf

% Change real and imaginary components to look normal
\renewcommand{\Re}{\operatorname{Re}}
\renewcommand{\Im}{\operatorname{Im}}

% Change QED symbol
% \renewcommand\qedsymbol{\emph{Do you believe me now!?}}
\renewcommand\qedsymbol{\emph{Quick maths}}

% Blackboard bold letters
\newcommand{\A}{\mathbb{A}}
\newcommand{\C}{\mathbb{C}}
\newcommand{\F}{\mathbb{F}}
\newcommand{\K}{\mathbb{K}}
\newcommand{\N}{\mathbb{N}}
\newcommand{\Q}{\mathbb{Q}}
\newcommand{\R}{\mathbb{R}}
\newcommand{\V}{\mathbb{V}}
\newcommand{\Z}{\mathbb{Z}}

% And, Or
\newcommand{\tor}{\text{ or }}
\newcommand{\tand}{\text{ and }}


% Theorem environment
\definecolor{theorem_color_background}{HTML}{F8E7EE}
\definecolor{theorem_color_frame}{HTML}{C73A71}
\definecolor{theorem_color_title}{HTML}{FFFFFF}
\newtcbtheorem[number within=section]{theorem}{Theorem}
{
  skin=enhanced,
  boxrule = 0.75mm,
  sharp corners,
  % Title settings
  attach boxed title to top left = {
      yshift*=-\tcboxedtitleheight/2,
      xshift = 10mm,
    },
  boxed title style = {colback=theorem_color_frame, sharp corners},
  % Adjust colors
  colback= theorem_color_background,
  colframe= theorem_color_frame,
  coltitle = theorem_color_title,
}{theorem}

% Proposition environment
\definecolor{proposition_color_background}{HTML}{FCF6EB}
\definecolor{proposition_color_frame}{HTML}{DCA323}
\definecolor{proposition_color_title}{HTML}{000000}
\newtcbtheorem[number within=section]{proposition}{Proposition}
{
  skin=enhanced,
  boxrule = 0.75mm,
  sharp corners,
  % Title settings
  attach boxed title to top left = {
      yshift*=-\tcboxedtitleheight/2,
      xshift = 10mm,
    },
  boxed title style = {colback=proposition_color_frame, sharp corners},
  % Adjust colors
  colback= proposition_color_background,
  colframe= proposition_color_frame,
  coltitle = proposition_color_title,
}{proposition}

% Lemma environment
\definecolor{lemma_color_background}{HTML}{DFF0DC}
\definecolor{lemma_color_frame}{HTML}{468C3B}
\definecolor{lemma_color_title}{HTML}{FFFFFF}
\newtcbtheorem[number within=section]{lemma}{Lemma}
{
  skin=enhanced,
  boxrule = 0.75mm,
  sharp corners,
  % Title settings
  attach boxed title to top left = {
      yshift*=-\tcboxedtitleheight/2,
      xshift = 10mm,
    },
  boxed title style = {colback=lemma_color_frame, sharp corners},
  % Adjust colors
  colback= lemma_color_background,
  colframe= lemma_color_frame,
  coltitle = lemma_color_title,
}{lemma}


% Fact environment
\definecolor{fact_color_background}{HTML}{DCE3FF}
\definecolor{fact_color_frame}{HTML}{3F69FF}
\definecolor{fact_color_title}{HTML}{FFFFFF}
\newtcbtheorem[number within=section]{fact}{Fact}
{
  skin=enhanced,
  boxrule = 0.75mm,
  sharp corners,
  % Title settings
  attach boxed title to top left = {
      yshift*=-\tcboxedtitleheight/2,
      xshift = 10mm,
    },
  boxed title style = {colback=fact_color_frame, sharp corners},
  % Adjust colors
  colback= fact_color_background,
  colframe= fact_color_frame,
  coltitle = fact_color_title,
}{fact}

% Problem environment
\definecolor{problem_color_background}{HTML}{FFDFCF}
\definecolor{problem_color_frame}{HTML}{FF5200}
\definecolor{problem_color_title}{HTML}{000000}
\newtcbtheorem[number within=section]{problem}{Problem}
{
  skin=enhanced,
  boxrule = 0.75mm,
  sharp corners,
  % Title settings
  attach boxed title to top left = {
      yshift*=-\tcboxedtitleheight/2,
      xshift = 10mm,
    },
  boxed title style = {colback=problem_color_frame, sharp corners},
  % Adjust colors
  colback= problem_color_background,
  colframe= problem_color_frame,
  coltitle = problem_color_title,
}{problem}

% Definition environment
\definecolor{definition_color_background}{HTML}{D6D6E0}
\definecolor{definition_color_frame}{HTML}{68668A}
\definecolor{definition_color_title}{HTML}{FFFFFF}
\newtcbtheorem[number within=section]{definition}{Definition}
{
  skin=enhanced,
  boxrule = 0.75mm,
  sharp corners,
  % Title settings
  attach boxed title to top left = {
      yshift*=-\tcboxedtitleheight/2,
      xshift = 10mm,
    },
  boxed title style = {colback=definition_color_frame, sharp corners},
  % Adjust colors
  colback= definition_color_background,
  colframe= definition_color_frame,
  coltitle = definition_color_title,
}{definition}

% Note environment
\definecolor{notation_color_background}{HTML}{F9D8DE}
\definecolor{notation_color_frame}{HTML}{F2ACB8}
\definecolor{notation_color_title}{HTML}{FFFFFF}
\newtcbtheorem[number within=section]{notation}{Notation}
{
  skin=enhanced,
  boxrule = 0.75mm,
  sharp corners,
  % Title settings
  attach boxed title to top left = {
      yshift*=-\tcboxedtitleheight/2,
      xshift = 10mm,
    },
  boxed title style = {colback=notation_color_frame, sharp corners},
  % Adjust colors
  colback= notation_color_background,
  colframe= notation_color_frame,
  coltitle = notation_color_title,
}{notation}

% Example environment
\definecolor{example_color_background}{HTML}{DDDCB6}
\definecolor{example_color_frame}{HTML}{BAB86B}
\definecolor{example_color_title}{HTML}{000000}
\newtcbtheorem[number within=section]{example}{Example}
{
  skin=enhanced,
  boxrule = 0.75mm,
  sharp corners,
  % Title settings
  attach boxed title to top left = {
      yshift*=-\tcboxedtitleheight/2,
      xshift = 10mm,
    },
  boxed title style = {colback=example_color_frame, sharp corners},
  % Adjust colors
  colback= example_color_background,
  colframe= example_color_frame,
  coltitle = example_color_title,
}{example}



\usepackage{enumitem}

\newcommand{\tp}{\mathcal{O}}
\newcommand{\pset}[1]{\mathcal{P}(#1)}
\newcommand{\defeq}{\stackrel{\text{def}}{=}}

\newcommand{\dom}{\operatorname{dom}}


\begin{document}

\begin{definition}{Topological Space}{}
    A \emph{topology} $\tp \subseteq \pset{M}$ on a set $M$ forms a \emph{topological space} $(M, \tp)$ if it satisfies the following,

    \begin{enumerate}[label=(\roman*)]
        \item $\emptyset, M \in \tp$.
        \item If $S \subseteq \tp$ then $\bigcup S \in \tp$.
        \item If $S_1, S_2 \in \tp$ then $S_1 \cap S_2 \in \tp$.
    \end{enumerate}
\end{definition}

\begin{terminology}{Open Set}{Open Set}
    Let $(M, \tp)$ be topological space, then a subset $U \subseteq M$ is called open if $U \in \tp$.
\end{terminology}

\begin{terminology}{Open Neighbourhood}{Open Neighbourhood}
    Let $(M, \tp)$ be topological space, then a subset $U \subseteq M$ is an open neighbourhood around $p \in M$ if it contains $p$ and is open.
    $$p \in U \in \tp$$
\end{terminology}

\begin{definition}{Continuous map}{Continuous map}
    Let $(M, \tp_M)$ and $(N, \tp_N)$ be topological spaces, then a map $f: M \to N$ is continuous at $p$ if:
    $$\forall U \in \tp_N: f(p) \in U \implies f^{\text{pre}}(U) \in \tp_M$$
\end{definition}

\begin{definition}{Fields}{Field}
    A \emph{field} $\left(K, +, \cdot\right)$ is a nonempty set $K$, along with two binary operations,
    addition $+: K \times K \to K$
    and multiplication $\cdot: K \times K \to K$,
    satisfying the following,

    \begin{enumerate}[label=(\roman*)]
        \item  For all $x, y, z \in K, (x + y) + z = x + (y + z) \tand (x \cdot y) \cdot z = x \cdot (y \cdot z)$.
        \item  For all $x, y \in K$, $x + y = y + x \tand x \cdot y = y \cdot x$.
        \item  There exists elements $0, 1 \in K$ such that for all $x \in K, x + 0 = x \tand x \cdot 1 = x$.
        \item  For all $x \in K$, there exists an element $-x \in K$ such that $x + (-x) = 0$, and if $x \neq 0$, there exists an element $x^{-1}$ such that $x \cdot x^{-1} = 1$.
        \item  For all $x, y, z \in K$, $x \cdot (y + z) = (x \cdot y) + (x \cdot z)$.
    \end{enumerate}
\end{definition}

\begin{definition}{Vecotor Spaces}{Vector Spaces}
    A \emph{Vector Space} $\left(V, +, \cdot\right)$ over a field $\left(K, +, \cdot\right)$ is a set $V$ along with two binary operations,
    vector addition $+: V \times V \to V$
    and s-multiplication $\cdot: K \times V \to V$,
    satisfying the following,

    \begin{enumerate}[label=(\roman*)]
        \item  For all vectors $v, w, u \in V$, $v + (w + u) = (v + w) + u$.
        \item  For all vectors $v, w \in V$, $v + w = w + v$.
        \item  There exists a vector $\vec{0} \in V$ such that $v + \vec{0} = v$ for all $v \in V$.
        \item  For every vector $v \in V$ there exists an element $-v$ such that $v + (-v) = \vec{0}$.
        \item  For all scalars $a, b \in K$ and vector $v \in V$, $a \cdot_V (b \cdot_V v) = (a \cdot_F b) \cdot_V v$.
        \item  For every scalar $a \in K$ and vectors $v, w \in V$, $a \cdot (v + w) = a \cdot v + a \cdot w$.
        \item  For all scalars $a, b \in K$ and vector $v\in V$, $(a + b) \cdot v = a \cdot v + b \cdot v$.
    \end{enumerate}
\end{definition}

\begin{definition}{The directional derivative}{The directional derivative}
    Let $f : \R^n \to \R^m$, for some $n, m \in \N$.
    Then we define the directional derivative of $f$ along $v \in \R^n$ as:
    $$\partial_v f : U \subseteq \R^n \to \R^m := p \mapsto \lim_{h \to 0} \frac{f(p + hv) - f(p)}{h}$$
\end{definition}

\begin{notation}{}{}
    \begin{enumerate}[label=(\roman*)]
        \item Let $f: \R \to \R^n$ then define $f' := \partial f := \partial_1 f$.
        \item Let $f: \R^n \to \R^m$ and $e_i$ be a basis for $\R^n$ then define $\partial_i f := \partial_{e_i} f$.
        \item Let $f: \R^n \to \R^m$ and $e_i$ be a basis for $\R^n$ then define $\frac{\partial f}{\partial e_i} := \partial_{e_i} f$.
    \end{enumerate}
\end{notation}{}{}



\begin{example}{Standerd topology on the Reals}{Standerd topology on the R}
    Let $\R$ be the set of real numbers, we define the standerd topology $\tp_s$ on $\R$.
    $$\tp_s = \{U \in \pset{\R} \mid \forall p  \in U : \exists a, b \in U : p \in (a,b) \tand (a,b) \subseteq U\}$$
\end{example}

\begin{definition}{}{}
    Let $(M, \tp_M)$ and $(N, \tp_N)$ be topological spaces, define the product topology $\tp_{M \times N}$ for the set $M \times N$,
    $$\tp_{M \times N}  := \{U \in \pset{\tp_M \times \tp_N} \mid \forall p \in U : \exists A \in \tp_M : \exists B \in \tp_N : p \in A \times B \subseteq U\}$$
\end{definition}



\begin{definition}{Norm Spaces}{Norm Spaces}
    A \emph{Norm Space} $\left(V, +, \cdot, \|\cdot\|\right)$ over a field $\left(K, +, \cdot\right)$
    is a vector space $\left(V, +, \cdot\right)$ over the field $\left(K, +, \cdot\right)$ along with a \emph{norm} $\|\cdot\| : V \to \R$  satisfying the following,

    \begin{enumerate}[label=(\roman*)]
        \item For every vector $v \in V$, $\|v\| \geq 0$.
        \item $\|v\| = 0$ if and only if $v = \vec{0}$.
        \item For every scalar $\lambda \in K$ and vector $v \in V$, $\|\lambda v\| = |\lambda|\|v\|$.
    \end{enumerate}
\end{definition}

\begin{definition}{Sequence}{Sequence}
    A \emph{Sequence} $a_n$ is map $a_n : \N \to X$ for some target $X$.
    A \emph{Finite Sequence} $a_n$ with a length $L \in \N$ is map $a_n : \{m \in \N \mid m \leq L\} \to X$ for some target $X$.
\end{definition}

\begin{definition}{Cauchy Sequence}{Cauchy Sequence}
    Let $\left(V, +, \cdot, \|\cdot\|\right)$ be normed space.
    A Sequence $a_n: \N \to V$ is \emph{Cauchy} if:
    $$\forall \varepsilon > 0 : \exists N : \forall m, n \geq N \implies \|x_n - x_m\| < \varepsilon.$$
    Meaning that elements in the sequence become closer and closer.
\end{definition}


\end{document}